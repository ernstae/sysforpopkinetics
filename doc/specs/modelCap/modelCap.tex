%***************************************************************
%
% File: modelCap.tex
%
%
% This document specifies the modeling capabilities for 
% the three tiered version of SPK.
%
% Author: Mitch Watrous
%
%***************************************************************

%---------------------------------------------------------------
%
% Preliminaries
%
%---------------------------------------------------------------

% Specifies the document style.
\documentclass{article}

% Horizontal and vertical offset of page relative to 1in
\hoffset        = -.5in
\voffset        = 0.0in

% Margins between offset and start of text
\topmargin      = .0in
\evensidemargin = .0in
\oddsidemargin  = .0in

% Amount of space for the header
\headheight     = .0in     % space for actual header
\headsep        = .0in     % space from bottom of header to beginning of text

% Amount of space for the footer
\footskip       = .5in     % distance from text to bottom of footer
%\footheight    = .5in    % space for actual footer

% Size of the actual text in the page
\textheight     = 10in
\textwidth      = 7.5in


%---------------------------------------------------------------
%
% Begin document
%
%---------------------------------------------------------------

% End of preamble and beginning of text.
\begin{document}

% large double spacing
\large
\baselineskip = 17pt

% Special macros
\newcommand{\R}{\mbox{\bf R}}
\newcommand{\Z}{\mbox{\bf Z}}
\newcommand{\N}{\mbox{\bf N}}
\newcommand{\E}{\mbox{\bf E}}
\newcommand{\V}{\mbox{\bf V}}
\newcommand{\Th}[1]{$#1$th}
\newcommand{\BS}{$\backslash$}
\newcommand{\Arg}[1]{\bigskip \noindent $#1$ \linebreak}
\newcommand{\TT}[1]{\bigskip \noindent {\tt #1 } \linebreak}
\newcommand{\rvec}{\mbox{rvec}}


%-----------------------------------------------------------------
%
% Title, table of contents, etc.
%
%-----------------------------------------------------------------

%-----------------------------------------------------------------
% Title
%-----------------------------------------------------------------

\title{Modeling Capabilities Specification}

\author{Mitch Watrous and Sachiko Honda}

\maketitle


%-----------------------------------------------------------------
% Table of contents
%-----------------------------------------------------------------

\tableofcontents

\newpage


%***************************************************************
%
\section{Introduction}
%
%***************************************************************

This document describes the modeling capabilities for the 
three-tiered version of the System for Population Kinetics (SPK).


%***************************************************************
%
\section{SPK Overview}
%
%***************************************************************

The three tiers of this system are
\begin{itemize}
  \item the NONMEM Model Design Agent (MDA),
  \item the Application Server for Population Kinetics (ASPK), and
  \item and the Computational Server for Population Kinetics (CSPK).
\end{itemize}


%***************************************************************
%
\section{Major Modeling Capabilities}
%
%***************************************************************

SPK will have the following major modeling capabilities:
\begin{itemize}
  \item individual level estimation,
  \item individual level simulation,
  \item population level estimation,
  \item population level simulation, and
  \item population level likelihood evaluation.
\end{itemize}


%***************************************************************
%
\section{Modeling Capability Extensions}
%
%***************************************************************

SPK will have some capabilities that go beyond those of NONMEM.

In particular, the modeling capabilities associated with the 
evaluation of population level likelihoods and with the 
calculation of some types of residuals are unique to SPK.
These capabilities are discussed in this section.


%-----------------------------------------------------------------
\subsection{Population Level Likelihood Evaluation}
%-----------------------------------------------------------------

SPK will allow for the evaluation of the population level likelihood
using Monte Carlo integration techniques.

%[Remove]====================================================
%
\begin{center}
\framebox{FINISH THIS.}
\end{center}
%
%[Remove]====================================================


%-----------------------------------------------------------------
\subsection{Residual Calculation}
%-----------------------------------------------------------------

Some of the residuals that SPK will calculate are different than those
that can be calculated using NONMEM.


%-----------------------------------------------------------------
\subsubsection{Extended Least Squares Individual Level Residuals}
%-----------------------------------------------------------------

Although the residuals discussed in this section are calculated in the
same way by SPK and NONMEM, this discussion is included here for
completeness.

When the extended least squares objective function is being used
to perform individual level estimation,
SPK's model for the $j^{\mbox{th}}$ value of an individual's data is
  \begin{equation}
    y_{(j)} = f_{(j)}(b) + e_{(j)} ,
  \end{equation}
where
  \begin{equation}
    e \sim \mbox{N}[ 0, R(b) ] ,
  \end{equation}
$b$ is the individual's vector of parameters, and $e$ is a vector of 
random variables that describe the measurement noise.
For a complete description of the models used at the individual 
level by SPK and NONMEM, see Appendix \ref{App:IndLevelModels}.

Let the vector of residuals for the individual be defined as the
differences between their data and the expected values for their
data as predicted by the model,
  \begin{eqnarray}
    r & = & y - \mbox{E}_{e} 
              \left[ f(b) + e \right] \nonumber \\
      & = & y - f(b) .
  \end{eqnarray}
The covariance matrix for the individual's data is
given by
  \begin{eqnarray}
    \mbox{cov}[y] & = & \mbox{cov}_{e}
          \left[ f(b) + e \right] \nonumber \\
        & = & R(b) .
  \end{eqnarray}
The vector of weighted residuals for the individual is
defined as 
  \begin{equation}
    w = \left\{ \mbox{cov}[y] \right\}^{-1/2} r,
  \end{equation}
where the term multiplying the residuals is the matrix square 
root of the inverse of the covariance.

The individual level residuals are calculated by evaluating the
preceding quantities at the optimal values for the individuals'
vectors of parameters that were determined during
individual level estimation, i.e., by setting $b = \hat{b}$.

The following table lists the residual related quantities, their
values as calculated by SPK, and their values as calculated by NONMEM.
In this case, the NONMEM values are calculated using its standard
individual estimation mode.

\begin{center}
\begin{tabular}{|p{0.75in}|p{3.25in}|p{1.1in}|p{0.85in}|p{1.0in}|}
\hline
\hline
  {\bf Quantity}
    & {\bf SPK Value}
    & {\bf Description}
    & {\bf Name}
    & {\bf NONMEM Value} \\
  \hline
  \hline
  $\mbox{E} \left[ f \right]^{\mbox{(ELS)}}$
    & $f(\hat{b})$
    & Predicted value for an individual's data.
    & PRED
    & Same \\
  \hline
  $r^{\mbox{(ELS)}}$
    & $y - f(\hat{b})$
    & Residuals for an individual.
    & RES 
    & Same \\
  \hline
  $w^{\mbox{(ELS)}}$
    & $\left\{ R(\hat{b}) \right\}^{-1/2}
        [ y - f(\hat{b}) ] $
    & Weighted residuals for an individual.
    & WRES 
    & Same \\
  \hline
  \hline
\end{tabular}
\end{center}


%-----------------------------------------------------------------
\subsubsection{Map Bayesian Individual Level Residuals}
%-----------------------------------------------------------------

When the Map Bayesian objective function is being used
to perform individual level estimation,
SPK's model for the $j^{\mbox{th}}$ value of an individual's data is
  \begin{equation}
    y_{(j)} = f_{(j)}(b) + e_{(j)} ,
  \end{equation}
where
  \begin{equation}
    b \sim \mbox{N}[ 0, D ] ,
  \end{equation}
  \begin{equation}
    e \sim \mbox{N}[ 0, R(b) ] ,
  \end{equation}
$b$ is the individual's vector of parameters, and $e$ is a vector of 
random variables that describe the measurement noise.
Note that $b$ is also a random variable in this case.
For a complete description of the models used at the individual 
level by SPK and NONMEM, see Appendix \ref{App:IndLevelModels}.

For this objective function, there is an additional set of residuals
that come from the terms that include the individual parameters $b$.
Let the vector of residuals for the individual parameters be defined as the
differences between zero and the expected values for the individual parameters,
  \begin{eqnarray}
    \rho & = & 0 - \mbox{E}_{b} 
              \left[ b \right] \nonumber \\
      & = & - b .
  \end{eqnarray}
The covariance matrix for the individual parameters is
  \begin{eqnarray}
    \mbox{cov}[b] & = & \mbox{cov}_{b}
          \left[ b \right] \nonumber \\
        & = & D .
  \end{eqnarray}
The vector of weighted individual parameter residuals is
defined as 
  \begin{equation}
    z = \left\{ \mbox{cov}[b] \right\}^{-1/2} \rho,
  \end{equation}
where the term multiplying the residuals is the matrix square 
root of the inverse of the covariance.

The individual level residuals are calculated by evaluating the
preceding quantities at the optimal values for the individuals'
vectors of parameters that were determined during
individual level estimation,$\hat{b}$.

The following table lists the residual related quantities, their
values as calculated by SPK, and their values as calculated by NONMEM.
In this case, the NONMEM values are calculated for all of the
individuals in the population using the POSTHOC option in the
\$ESTIMATION command.

\begin{center}
\begin{tabular}{|p{0.75in}|p{3.25in}|p{1.1in}|p{0.85in}|p{1.0in}|}
\hline
\hline
  {\bf Quantity}
    & {\bf SPK Value}
    & {\bf Description}
    & {\bf Name}
    & {\bf NONMEM Value} \\
  \hline
  \hline
  $\mbox{E} \left[ f \right]^{\mbox{(MB)}}$
    & $f(\hat{b})$
    & Predicted value for an individual's data.
    & PRED 
    & Same \\
  \hline
  $r^{\mbox{(MB)}}$
    & $y - f(\hat{b})$
    & Residuals for an individual.
    & RES 
    & Same \\
  \hline
  $w^{\mbox{(MB)}}$
    & $\left\{ R(\hat{b}) \right\}^{-1/2}
        [ y - f(\hat{b}) ] $
    & Weighted residuals for an individual.
    & WRES 
    & Same \\
  \hline
  $\rho^{\mbox{(MB)}}$
    & $-\hat{b}$
    & Individual parameter residuals.
    & ETARES
    & Not available in NONMEM. \\
  \hline
  $z^{\mbox{(MB)}}$
    & $\left\{ D \right\}^{-1/2} ( - \hat{b} )$
    & Weighted individual parameter residuals.
    & WETARES
    & Not available in NONMEM. \\
  \hline
  \hline
\end{tabular}
\end{center}


%-----------------------------------------------------------------
\subsubsection{Population Level Residuals Background}
%-----------------------------------------------------------------

SPK's mixed effects model for the $j^{\mbox{th}}$ value of the 
$i^{\mbox{th}}$ individual's data is
  \begin{equation}
    y_{i(j)} = f_{i(j)}(\alpha, b_i) + e_{i(j)} ,
  \end{equation}
where
  \begin{equation}
    b_i \sim \mbox{N}[ 0, D(\alpha) ] ,
  \end{equation}
  \begin{equation}
    e_i \sim \mbox{N}[ 0, R_i(\alpha, b_i) ] ,
  \end{equation}
$\alpha$ is a vector of fixed effects parameters, 
$b_i$ is the individual's vector of random effects parameters,
and $e_i$ is a vector of random variables that describe the
measurement noise.
For a complete description of the models used at the population 
level by SPK and NONMEM, see Appendix \ref{App:PopLevelModels}.

Let the vector of residuals for the $i^{\mbox{th}}$ individual be
defined as the differences between their data and the expected values 
for their data as predicted by the model,
  \begin{eqnarray}
    r_i & = & y_i - \mbox{E}_{b,e} 
                \left[ f_i(\alpha, b_i) + e_i \right] \nonumber \\
        & = & y_i - \mbox{E}_{b} \left[ f_i(\alpha, b_i) \right] .
  \end{eqnarray}
The covariance matrix for the $i^{\mbox{th}}$ individual's data is
given by
  \begin{eqnarray}
    \mbox{cov}[y_i] & = & \mbox{cov}_{b,e}
          \left[ f_i(\alpha, b_i) + e_i \right] \nonumber \\
        & = & \mbox{cov}_{b} \left[ f_i(\alpha, b_i) \right]
                  + R_i(\alpha, b_i) .
  \end{eqnarray}
The vector of weighted residuals for the $i^{\mbox{th}}$ individual is
defined as 
  \begin{equation}
    w_i = \left\{ \mbox{cov}[y_i] \right\}^{-1/2} r_i,
  \end{equation}
where the term multiplying the residuals is the matrix square 
root of the inverse of the covariance.

In order to calculate the residuals and weighted residuals, SPK makes
some approximations that are based on the derivations and discussion
in Sections (6.2) and (6.3) of Davidian and Giltinan\footnote{
  Davidian M. and Giltinan D. M. (1998) 
  {\em Nonlinear Models for Repeated Measurement Data}, Chapman \&
  Hall/CRC, Boca Raton, Florida, pp. 152-66.
}.

To be specific, SPK makes the following Taylor series approximations 
about the value $b^{\ast}_i$,
  \begin{eqnarray}
    f_i(\alpha, b_i) & \approx &  f_i(\alpha, b^{\ast}_i)
      + \partial_b f_i(\alpha, b^{\ast}_i) ( b_i - b^{\ast}_i ) , \\
    R_i(\alpha, b_i) & \approx &  R_i(\alpha, b^{\ast}_i) .
  \end{eqnarray}
The residuals are therefore approximated by
  \begin{equation}
    r_i \approx y_i - f_i(\alpha, b^{\ast}_i)
      + \partial_b f_i(\alpha, b^{\ast}_i) \; b^{\ast}_i .
  \end{equation}
The covariance of the individual's data is approximated by
  \begin{eqnarray}
    \mbox{cov}[y_i] \approx \partial_b f_i(\alpha, b^{\ast}_i) \;
        D(\alpha) \; \partial_b f_i(\alpha, b^{\ast}_i)^T
      + R_i(\alpha, b^{\ast}_i) .
  \end{eqnarray}

As a diagnostic, SPK will also provide the weighted individual 
parameter residuals at the population level.
Let the vector of residuals for the $i^{\mbox{th}}$ individual's 
parameters be defined as the differences between zero and the 
values for their individual parameters,
  \begin{equation}
    \rho_i = - b_i .
  \end{equation}
The covariance matrix for the individual parameters is
  \begin{eqnarray}
    \mbox{cov}[b_i] & = & \mbox{cov}_{b}
          \left[ b_i \right] \nonumber \\
        & = & D(\alpha) .
  \end{eqnarray}
The vector of weighted individual parameter residuals is
defined as 
  \begin{equation}
    z_i = \left\{ \mbox{cov}[b_i] \right\}^{-1/2} \rho_i,
  \end{equation}
where the term multiplying the residuals is the matrix square 
root of the inverse of the covariance.


%-----------------------------------------------------------------
\subsubsection{First Order Population Residuals}
%-----------------------------------------------------------------

The first class of population residuals are calculated by evaluating the
preceding approximations at the mean values for the individuals'
vectors of random effects parameters, i.e.,
by setting $b^{\ast}_i = 0$ for all of the individuals.
This corresponds to the approximations made for the First Order (FO)
objective function.

The following table lists the residual related quantities, their
values as calculated by SPK, and their values as calculated by NONMEM.
In this case, the NONMEM values are calculated using 
if First Order (FO) Method in the \$ESTIMATION command without 
the POSTHOC option.

\begin{center}
\begin{tabular}{|p{0.75in}|p{3.25in}|p{1.1in}|p{0.85in}|p{1.0in}|}
\hline
\hline
  {\bf Quantity}
    & {\bf SPK Value}
    & {\bf Description}
    & {\bf Name}
    & {\bf NONMEM Value} \\
  \hline
  \hline
  $\mbox{E} \left[ f_i \right]^{\mbox{(FO)}}$
    & $f_i(\alpha, 0)$
    & Approximate predicted value for an individual's data.
    & PRED 
    & Same \\
  \hline
  $r^{\mbox{(FO)}}_i$
    & $y_i - f_i(\alpha, 0)$
    & Approximate residuals for an individual.
    & RES 
    & Same \\
  \hline
  $w^{\mbox{(FO)}}_i$
    & $\begin{array}{l}
        \left\{ \partial_b f_i(\alpha, 0) \;
          D(\alpha) \; \partial_b f_i(\alpha, 0)^T
          + R_i(\alpha, 0) \right\}^{-1/2} \\
        \mbox{\hspace{0.5in}}
        \ast [ y_i - f_i(\alpha, 0) ]
      \end{array} $
    & Approximate weighted residuals for an individual.
    & WRES 
    & Same \\
  \hline
  $\rho^{\mbox{(FO)}}_i$
    & $0$
    & Individual parameter residuals for an individual.
    & ETARES
    & Not available in NONMEM. \\
  \hline
  $z^{\mbox{(FO)}}_i$
    & $0$
    & Weighted individual parameter residuals for an individual.
    & WETARES
    & Not available in NONMEM. \\
  \hline
  \hline
\end{tabular}
\end{center}


%-----------------------------------------------------------------
\subsubsection{Conditional Population Residuals}
%-----------------------------------------------------------------

The second class of population residuals are calculated by evaluating the
preceding approximations at the values determined for the individuals'
vectors of random effects parameters during one of the conditional
estimation methods, i.e.,  by setting $b^{\ast}_i = \hat{b}_i$ 
for all of the individuals.
The conditional estimation methods in SPK use either the Expected Hessian
objective function, which is equivalent to NONMEM'S First Order
Conditional Estimation (FOCE) method, or the Laplace
objective, which is equivalent to NONMEM's Laplace method.

The following table lists the residual related quantities, their
values as calculated by SPK, and their values as calculated by NONMEM.
In this case, the NONMEM values are calculated using 
one of NONMEM's conditional estimation that are available in the
\$ESTIMATION command.

\begin{center}
\begin{tabular}{|p{0.75in}|p{3.25in}|p{1.1in}|p{0.85in}|p{1.0in}|}
\hline
\hline
  {\bf Quantity}
    & {\bf SPK Value}
    & {\bf Description}
    & {\bf Name}
    & {\bf NONMEM Value} \\
  \hline
  \hline
  $\mbox{E} \left[ f_i \right]^{\mbox{(Cond)}}$
    & $f_i(\alpha, \hat{b}_i) - \partial_b f_i(\alpha, \hat{b}_i) \; \hat{b}_i$
    & Approximate predicted value for an individual's data.
    & PRED 
    & Same as FO values, i.e., $b^{\ast}_i = 0$ for all individuals\\
  \hline
  $r^{\mbox{(Cond)}}_i$
    & $y_i - f_i(\alpha, \hat{b}_i) + \partial_b f_i(\alpha, \hat{b}_i) \; \hat{b}_i$
    & Approximate residuals for an individual.
    & RES 
    & Same as FO values, i.e., $b^{\ast}_i = 0$ for all individuals\\
  \hline
  $w^{\mbox{(Cond)}}_i$
    & $\begin{array}{l}
        \left\{ \partial_b f_i(\alpha, \hat{b}_i) \;
          D(\alpha) \; \partial_b f_i(\alpha, \hat{b}_i)^T
          + R_i(\alpha, \hat{b}_i) \right\}^{-1/2} \\
        \mbox{\hspace{0.5in}}
        \ast [ y_i - f_i(\alpha, \hat{b}_i) 
           + \partial_b f_i(\alpha, \hat{b}_i) \; \hat{b}_i ]
      \end{array} $
    & Approximate weighted residuals for an individual.
    & WRES 
    & Same as FO values, i.e., $b^{\ast}_i = 0$ for all individuals\\
  \hline
  $\rho^{\mbox{(Cond)}}_i$
    & $-\hat{b_i}$
    & Individual parameter residuals for an individual.
    & ETARES
    & Not available in NONMEM. \\
  \hline
  $z^{\mbox{(Cond)}}_i$
    & $\left\{ D(\alpha) \right\}^{-1/2} ( - \hat{b_i} )$
    & Weighted individual parameter residuals for an individual.
    & WETARES
    & Not available in NONMEM. \\
  \hline
  \hline
\end{tabular}
\end{center}


%-----------------------------------------------------------------
\subsection{Other Extensions}
%-----------------------------------------------------------------

%[Remove]====================================================
%
\begin{center}
\framebox{ARE THERE OTHERS?}
\end{center}
%
%[Remove]====================================================



%***************************************************************
%
\section{Modeling Capability Restrictions}
%
%***************************************************************

SPK will not support all of NONMEM's modeling capabilities.

In particular, the modeling capabilities associated with some 
of the NM-TRAN records and keywords will be restricted.
These restriction are discussed in this section.


%-----------------------------------------------------------------
\subsection{Data Item Types (\$INPUT)}
%-----------------------------------------------------------------

The only reserved labels for the \$INPUT record that SPK will
recognize as having special meanings are the DV, EVID, and
ID labels.


%-----------------------------------------------------------------
\subsubsection{Reserved Labels not Recognized by SPK}
%-----------------------------------------------------------------

The other reserved labels
(L1, L2, MDV, RAW\_, MRG\_, TIME, DATE, DAT1, DAT2, DAT3, 
DROP, SKIP, AMT, RATE, SS, II, ADDL, CMT, PCMT, CALL, CONT)
can still be used as valid 
data item types in the \$INPUT record.
If one of these reserved labels appears in the \$INPUT record, 
then it may  be referenced within the \$PRED block in the usual way.

However, SPK will not recognize it as having special meaning and will
not consider it to be different than any other user defined label.
It will be the responsibility of the expressions in the \$PRED block to 
perform the tasks corresponding to these reserved data item labels.
These tasks are usually performed by NM-TRAN, NONMEM, and the
ADVAN library.


%-----------------------------------------------------------------
\subsubsection{Dependent Variable Data Item (DV)}
%-----------------------------------------------------------------

The dependent variable (DV) data item indicates which column in 
the data file contains the observed values.


%-----------------------------------------------------------------
\subsubsection{Event Identification Data Item (EVID)}
%-----------------------------------------------------------------

The event identification (EVID) data item is used to indicate which
column in the data file contains the EVID values.
These values describe the kind of event: observation, dose, 
other-type, reset, or reset-and-dose.
The only value in the EVID column that will have any special meaning
within SPK is EVID = 0, which corresponds to observation events.

Note that if every event is an observation event, then
the EVID column can be left out of the data file and the EVID label
does not need to be included in the \$INPUT record.


%-----------------------------------------------------------------
\subsubsection{Identification Data Item (ID)}
%-----------------------------------------------------------------

The identification (ID) data item indicates which column in 
the data file contains the ID values.
A change in the ID value indicates the start of a new
individual record.


%-----------------------------------------------------------------
\subsection{Data Sets (\$DATA)}
%-----------------------------------------------------------------

The NONMEM format data files that are supported will have some 
restrictions.


%-----------------------------------------------------------------
\subsubsection{Event Identification Data Items (EVID)}
%-----------------------------------------------------------------

The expressions from the \$PRED block will be evaluated for all
of the data values in the file.
But, only observation events (EVID = 0) will be associated
with dependent variable (DV) values, i.e., with $y_{i(j))}$.
If the data set does not contain an EVID column, then every
event will be assumed to be an observation event.


%-----------------------------------------------------------------
\subsubsection{Steady State Data Items (SS)}
%-----------------------------------------------------------------

No steady state (SS) data items will be allowed.


%-----------------------------------------------------------------
\subsection{Model Specification via the PRED Block (\$PRED)}
%-----------------------------------------------------------------

The specification of general models using the \$PRED block will
be allowed.
See Appendix \ref{App:CppVersionOfPred} for an example C++ 
implementation of the expressions in this block.


%-----------------------------------------------------------------
\subsubsection{Variable Naming Restrictions}
%-----------------------------------------------------------------

Variable names beginning with the characters ``spk\_'' are restricted
for use within SPK.


%-----------------------------------------------------------------
\subsubsection{Special NONMEM-PREDPP Arguments (NEWIND and ICALL)}
%-----------------------------------------------------------------

SPK will not support the NEWIND or ICALL arguments.


%-----------------------------------------------------------------
\subsubsection{Variables from NONMEM-PREDPP Common Blocks}
%-----------------------------------------------------------------

SPK will not allow variables from NONMEM-PREDPP common blocks to be
accessed in the \$PK block.


%-----------------------------------------------------------------
\subsubsection{Controlling the Calling Frequency (CALLFL)}
%-----------------------------------------------------------------

SPK will not allow the calling frequency to be set via NM-TRAN
pseudo assignment statements or via the CALLFL parameter.


%-----------------------------------------------------------------
\subsubsection{First-order Approximation for the Intraindividual Error Model
  during Population Level Estimation}
%-----------------------------------------------------------------

Because NM-TRAN makes a first-order approximation for the 
intraindividual error model during population level 
estimation\footnote{
  {\em NONMEM User's Guide - Part V}, NONMEM Project Group, 
  p. 81, 1994.
},
the corresponding SPK models generated by ASPK will do the same.

To be specific, let the model for 
the $j^{\mbox{th}}$ value of the $i^{\mbox{th}}$ individual's data
be expressed as the following functional:
  \begin{equation}
    y_{i(j)} = y_{i(j)}\left( \rule{0.0in}{0.15in}
      f(x_{ij}, \theta, \eta_i), x_{ij}, \theta, \eta_i, \epsilon_i \right) .
  \end{equation}
where $f(x_{ij}, \theta, \eta_i)$ is the mean or expected value 
for the data,
$x_{ij}$ is a vector of known quantities for the individual
such as times and covariates,
$\theta$ is a vector of fixed effects parameters,
$\eta_i$ is the individual's vector of random effects parameters,
and $\epsilon_i$ is a vector of random variables that appears in
NONMEM'S intraindividual error model.
For a complete description of the models used at the population 
level by SPK and NONMEM, see Appendix \ref{App:PopLevelModels}.

During population level estimation this functional is approximated by
  \begin{eqnarray}
    \lefteqn{ \widetilde{y}_{i(j)} \left( \rule{0.0in}{0.15in} f(x_{ij}, \theta, \eta_i), 
        x_{ij}, \theta, \eta_i, \epsilon_i \right) } 
        \nonumber \\
    & & = y_{i(j)}\left( \rule{0.0in}{0.15in} f(x_{ij}, \theta, \eta_i), 
        x_{ij}, \theta, \eta_i, \epsilon_i \right)
        \left| \!
          \begin{array}{l}
            \\
            {\scriptstyle \epsilon_i = 0}
          \end{array}
        \right.
      + \sum_{m=1}^{n_{\Sigma}} \left[ \partial_{\epsilon}^{(m)} 
        y_{i(j)}\left( \rule{0.0in}{0.15in} f(x_{ij}, \theta, \eta_i), 
        x_{ij}, \theta, \eta_i, \epsilon_i \right)
        \left| \!
          \begin{array}{l}
            \\
            {\scriptstyle \epsilon_i = 0}
          \end{array}
        \right.
        \right] \epsilon_{i(m)} .
  \end{eqnarray}
Define the derivative of the functional evaluated at $\epsilon_i = 0$ as
  \begin{equation}
    h_{(m)}\left( \rule{0.0in}{0.15in} f(x_{ij}, \theta, \eta_i), 
        x_{ij}, \theta, \eta_i \right)
      = \partial_{\epsilon}^{(m)} 
        y_{i(j)}\left( \rule{0.0in}{0.15in} f(x_{ij}, \theta, \eta_i), 
        x_{ij}, \theta, \eta_i, \epsilon_i \right)
        \left| \!
          \begin{array}{l}
            \\
            {\scriptstyle \epsilon_i = 0}
          \end{array}
        \right. .
  \end{equation}
Then, the covariance of the $j^{\mbox{th}}$ and $k^{\mbox{th}}$ values of 
the $i^{\mbox{th}}$ individual's data is given by
  \begin{eqnarray}
    \mbox{cov} [ \tilde{y}_{i(j)} , \tilde{y}_{i(k)} ]
    & = &  \mbox{cov} \left[
      \sum_{m=1}^{n_{\Sigma}} h_{(m)}\left( \rule{0.0in}{0.15in} 
        f(x_{ij}, \theta, \eta_i), x_{ij}, \theta, \eta_i \right) \epsilon_{i(m)},
      \sum_{n=1}^{n_{\Sigma}} h_{(n)}\left( \rule{0.0in}{0.15in} 
        f(x_{ik}, \theta, \eta_i), x_{ik}, \theta, \eta_i \right) \epsilon_{i(n)}
      \right] \\
    & = & \sum_{m=1}^{n_{\Sigma}} \sum_{n=1}^{n_{\Sigma}} \left[
      h_{(m)}\left( \rule{0.0in}{0.15in} f(x_{ij}, \theta, \eta_i),
        x_{ij}, \theta, \eta_i \right) 
      \right]
      \mbox{cov} [ \epsilon_{i(m)} , \epsilon_{i(n)} ]
      \left[
      h_{(n)}\left( \rule{0.0in}{0.15in} f(x_{ik}, \theta, \eta_i), 
        x_{ik}, \theta, \eta_i \right)
      \right]
    \\
    & = & \sum_{m=1}^{n_{\Sigma}} \sum_{n=1}^{n_{\Sigma}} \left[
      h_{(m)}\left( \rule{0.0in}{0.15in} f(x_{ij}, \theta, \eta_i),
        x_{ij}, \theta, \eta_i \right) 
      \right]
      \Sigma_{(m,n)}
      \left[
      h_{(n)}\left( \rule{0.0in}{0.15in} f(x_{ik}, \theta, \eta_i),
        x_{ik}, \theta, \eta_i \right)
     \right]
  \end{eqnarray}
This corresponds to $R_{i(j,k)}$ in SPK notation.

If the intraindividual error model can be expressed in traditional
NONMEM form, 
  \begin{equation}
    y_{i(j)} = f(x_{ij}, \theta, \eta_i) 
      + \sum_{m=1}^{n_{\Sigma}}
      h_{(m)}\left( \rule{0.0in}{0.15in} f(x_{ij}, \theta, \eta_i), 
        x_{ik}, \theta, \eta_i \right) \epsilon_{i(m)} ,
  \end{equation}
then 
  \begin{equation}
    \widetilde{y}_{i(j)} = y_{i(j)} ,
  \end{equation}
and there is no difference between the approximate and exact
intraindividual error models.

If the intraindividual error model is not linear
in $\epsilon$, however, the two forms are not equal.
For example, in the case of an exponential error model with a single 
$\epsilon$ component, 
  \begin{equation}
    y_{i(j)} = f(x_{ij}, \theta, \eta_i) \exp[ \epsilon_{i(1)} ] ,
  \end{equation}
which implies that
  \begin{eqnarray}
    \widetilde{y}_{i(j)} 
      & = & f(x_{ij}, \theta, \eta_i) 
        + f(x_{ij}, \theta, \eta_i) \epsilon_{i(1)} \\
      & \neq & y_{i(j)} ,
  \end{eqnarray}
and the intraindividual error models are therefore different.

Note that as a result of this first-order approximation
an exponential model is equivalent to a constant 
coefficient of variation (CCV) model during population level 
estimation.


%-----------------------------------------------------------------
\subsection{Pharmacokinetic Model Specification (\$SUBROUTINE)}
%-----------------------------------------------------------------

The specification of models using the \$SUBROUTINE block will not be allowed.


%-----------------------------------------------------------------
\subsubsection{Pharmacokinetic Library Subroutines (ADVAN)}
%-----------------------------------------------------------------

SPK will not support any of the models from the ADVAN pharmacokinetic
library.


%-----------------------------------------------------------------
\subsubsection{Translator Routines (TRANS)}
%-----------------------------------------------------------------

SPK will not support any of the parameterizations from the TRANS 
parameter translation library.


%-----------------------------------------------------------------
\subsection{Pharmacokinetic Parameter Models (\$PK)}
%-----------------------------------------------------------------

The methods for specification of pharmacokinetic parameter models will
be restricted.


%-----------------------------------------------------------------
\subsubsection{No Support for \$PK Block}
%-----------------------------------------------------------------

SPK will not support the specification of pharmacokinetic parameter
models using the \$PK block.


%-----------------------------------------------------------------
\subsubsection{Specifiying Pharmacokinetic Parameter Models using \$PRED}
%-----------------------------------------------------------------

The pharmacokinetic parameter model must be specified within the
\$PRED block.


%-----------------------------------------------------------------
\subsection{Initial Estimates and Bounds for Theta (\$THETA)}
%-----------------------------------------------------------------

The capabilities of the Theta vectors that are supported will 
have some restrictions.


%-----------------------------------------------------------------
\subsubsection{Unbounded Upper or Lower Limits}
%-----------------------------------------------------------------

SPK will not allow unbounded lower or upper limits, which are
represented in the \$THETA record as -1000000 or 1000000.


%-----------------------------------------------------------------
\subsection{Initial Estimates for Omega (\$OMEGA)}
%-----------------------------------------------------------------

The capabilities of the Omega matrices that are supported will 
have some restrictions.


%-----------------------------------------------------------------
\subsubsection{Diagonal or Full Matrices Only}
%-----------------------------------------------------------------

SPK will only support diagonal or full Omega matrices.


%-----------------------------------------------------------------
\subsubsection{Only One Block for Nondiagonal Matrices (BLOCK)}
%-----------------------------------------------------------------

SPK will only support nondiagonal Omega matrices that 
have a single block.


%-----------------------------------------------------------------
\subsection{Intraindividual Error Models (\$ERROR)}
%-----------------------------------------------------------------

The methods for specification of intraindividual error models will
be restricted.


%-----------------------------------------------------------------
\subsubsection{No Support for \$ERROR Block}
%-----------------------------------------------------------------

SPK will not support the specification of intraindividual error
models using the \$ERROR block.


%-----------------------------------------------------------------
\subsubsection{Specifiying Intraindividual Error Models using \$PRED}
%-----------------------------------------------------------------

The intraindividual error model must be specified within the
\$PRED block.


%-----------------------------------------------------------------
\subsection{Initial Estimates for Sigma (\$SIGMA)}
%-----------------------------------------------------------------

The capabilities of the Sigma matrices that are supported will 
have some restrictions.


%-----------------------------------------------------------------
\subsubsection{Diagonal or Full Matrices Only}
%-----------------------------------------------------------------

SPK will only support diagonal or full Sigma matrices.


%-----------------------------------------------------------------
\subsubsection{Only One Block for Nondiagonal Matrices (BLOCK)}
%-----------------------------------------------------------------

SPK will only support nondiagonal Sigma matrices that 
have a single block.


%-----------------------------------------------------------------
\subsection{Instructions for the Estimation Step (\$ESTIMATION)}
%-----------------------------------------------------------------

The support SPK will give to some of NONMEM's more complex parameter
estimation capabilities will be limited.


%-----------------------------------------------------------------
\subsubsection{Continue Estimation when Theta is Nonphysiological (NOABORT)}
%-----------------------------------------------------------------

SPK will not support this.


%-----------------------------------------------------------------
\subsubsection{Conditional Likelihood for Noncontinuous Observed
  Responses (LIKELIHOOD)}
%-----------------------------------------------------------------

SPK will not support this.


%-----------------------------------------------------------------
\subsubsection{Constrain Average Eta Estimates to be Close to Zero (CENTERING)}
%-----------------------------------------------------------------

SPK will not support this.


%-----------------------------------------------------------------
\subsubsection{Request Second Eta Derivatives to no be Numerical (NONUMERICAL)}
%-----------------------------------------------------------------

SPK will not support this.


%-----------------------------------------------------------------
\subsubsection{Fix Selected Etas to be Zero (ZERO)}
%-----------------------------------------------------------------

SPK will not support this.


%---------------------------------------------------------------
%
% Appendices
%
%---------------------------------------------------------------

\appendix
\newpage


%***************************************************************
%
\section{Appendix: The C++ Equivalent of the \$PRED Block}
%
%***************************************************************

\label{App:CppVersionOfPred}

The following C++ function is an example implementation of
the expressions in the NM-TRAN \$PRED block for the CONTROL4
Example
\footnote{
  {\em NONMEM User's Guide - Part VIII}, NONMEM Project Group, 
  pp. 306, 1998.
} 

This function will be generated by the Application Server for
Population Kinetics (ASPK) from the expressions in the \$PRED block.
Note that this function interface and example implementation will serve as
preliminary specifications for the function.

\begin{quotation}
\noindent
\begin{verbatim}
#include <libspkcompiler/nonmem.h>
#include "IndData.h"
namespace {
IndDataSet spk_all;
double cl;
double cp;
double d;
double dose;
double dv;
double e;
double evid;
double ka;
double ke;
double mdv;
double spk_time;
double wt;
};
template< class Type >
bool evalPred( const Type* const theta, 
               int         spk_nTheta, 
               const Type* const eta, 
               int         spk_nEta, 
               const Type* const eps, 
               int         spk_nEps, 
               int         spk_i, 
               int         spk_j, 
               double      &f, 
               double      &y )
{
cp = spk_all[spk_i].cp[spk_j];
dv = spk_all[spk_i].dv[spk_j];
dose = spk_all[spk_i].dose[spk_j];
evid = spk_all[spk_i].evid[spk_j];
mdv = spk_all[spk_i].mdv[spk_j];
spk_time = spk_all[spk_i].time[spk_j];
wt = spk_all[spk_i].wt[spk_j];
//============================================
//   User's Code Begin
//--------------------------------------------
if( dose != 0 ){
  ds = dose * wt;
  w = wt;
}
ka = theta[0] + eta[0];
ke = theta[1] + eta[1];
cl = theta[2] * w + eta[2];
d = exp(-ke * spk_time) - exp(-ka * spk_time);
e = cl * (ka - ke);
f = ds * ke * ka / e * d;
y = f + eps[0];
y = f + eps[0];
//--------------------------------------------
//   End of User's Code
//============================================
if( spk_all[spk_i].evid[spk_j] == nonmem::EVID_OBSERVATION )
   return true;
return false;
}
\end{verbatim}
\end{quotation}

\newpage


%***************************************************************
%
\section{Appendix: Individual Level Models}
%
%***************************************************************

\label{App:IndLevelModels}.

This appendix describes the models used at the individual level 
by SPK and NONMEM.


%-----------------------------------------------------------------
\subsection{SPK Model for an Individual's Data}
%-----------------------------------------------------------------

SPK's model for the $j^{\mbox{th}}$ value of an individual's
data is 
  \begin{equation}
    y_{(j)} = f_{(j)}(b) + e_{(j)} ,
  \end{equation}
where
  \begin{equation}
    R_{(j,k)}(b) = \mbox{cov}[e_{(j)},e_{(k)}] ,
  \end{equation}
and $b$ is the individual's vector of parameters.
If the MAP Bayesian objective function is used, then 
  \begin{equation}
    D_{(p,q)} = \mbox{cov}[b_{(p)},b_{(q)}] ,
  \end{equation}
is the covariance of the individual's parameters.

The output of SPK's individual level estimation is
  \begin{equation}
    b^{\mbox{Out}} ,
  \end{equation}
which is an estimate for the optimal value for the individual's
parameters.


%-----------------------------------------------------------------
\subsection{NONMEM Model for an Individual's Data}
%-----------------------------------------------------------------

NONMEM's model 
\footnote{
  {\em NONMEM User's Guide - Part V}, NONMEM Project Group, 
  pp. 23-31, 1994.
} 
for the $j^{\mbox{th}}$ value of an individual's data is 
  \begin{equation}
    y_{(j)} = f(x_{j}, \theta) + 
      \sum_{m=1}^{n_{\Omega}} h_{(m)}(x_{j}, \theta) \eta_{(m)} .
  \end{equation}
where
  \begin{equation}
    \Omega_{(m,n)} = \mbox{cov}[\eta_{(m)},\eta_{(n)}] ,
  \end{equation}
$\theta$ and $\Omega$ are a vector and diagonal matrix of 
the individual's parameters, 
$x_{j}$ is a vector of known quantities for the individual
such as times and covariates,
and $n_{\Omega}$ is the number of elements along the diagonal
of $\Omega$.
NONMEM does not support the MAP Bayesian objective function
at the individual level.

The output of NONMEM's individual level estimation is
  \begin{equation}
    \left\{ \theta^{\mbox{Out}}, \Omega^{\mbox{Out}} \right\} ,
  \end{equation}
which are estimates for the optimal values for the individual's
parameters.


%-----------------------------------------------------------------
\subsection{Comparison of the Models for an Individual's Data}
%-----------------------------------------------------------------

In these two formulations, $b$ and $\eta$ are not equivalent,
and  there is no relationship between the covariance matrices 
$D$ and $\Omega$.

Although $e$ and $\eta$ are also not equivalent, 
the covariance matrices $R(b)$ and 
$\Omega$ are related as follows,
  \begin{eqnarray*}
    R_{(j,k)}(b) & = &  \mbox{cov} \left[
      \sum_{m=1}^{n_{\Omega}} h_{(m)}(x_{j}, \theta) \eta_{(m)},
      \sum_{n=1}^{n_{\Omega}} h_{(n)}(x_{k}, \theta) \eta_{(n)} \right] \\
    & = & \sum_{m=1}^{n_{\Omega}} \sum_{n=1}^{n_{\Omega}} 
      h_{(m)}(x_{j}, \theta) 
      \mbox{cov} [ \eta_{(m)} , \eta_{(n)} ]
      h_{(n)}(x_{k}, \theta)  \\
    & = & \sum_{m=1}^{n_{\Omega}} \sum_{n=1}^{n_{\Omega}} 
      h_{(m)}(x_{j}, \theta) 
      \Omega_{(m,n)}
      h_{(n)}(x_{k}, \theta) \\
    & = & \sum_{m=1}^{n_{\Omega}}
      h_{(m)}(x_{j}, \theta) 
      \Omega_{(m,m)}
      h_{(m)}(x_{k}, \theta) .
  \end{eqnarray*}


\newpage


%***************************************************************
%
\section{Appendix: Population Level Models}
%
%***************************************************************

\label{App:PopLevelModels}.

This appendix describes the mixed effects models used at the
population level by SPK and NONMEM. 


%-----------------------------------------------------------------
\subsection{SPK Mixed Effects Model}
%-----------------------------------------------------------------

SPK's mixed effects model for the $j^{\mbox{th}}$ value of the
$i^{\mbox{th}}$ individual's data is 
  \begin{equation}
    y_{i(j)} = f_{i(j)}(\alpha, b_i) + e_{i(j)} ,
  \end{equation}
where
  \begin{equation}
    R_{i(j,k)}(\alpha, b_i) = \mbox{cov}[e_{i(j)},e_{i(k)}] ,
  \end{equation}
  \begin{equation}
    D_{(p,q)}(\alpha) = \mbox{cov}[b_{i(p)},b_{i(q)}] ,
  \end{equation}
$\alpha$ is a vector of fixed effects parameters, 
and $b_i$ is the individual's vector of random effects parameters.

The output of SPK's population level estimation is
  \begin{equation}
    \left\{ \alpha^{\mbox{Out}}, 
      \{ b_i^{\mbox{Out}} ( \alpha^{\mbox{Out}} ) \} \right\} ,
  \end{equation}
which are estimates for the optimal values for the fixed 
effects parameters along with estimate for the optimal values
for each individual's random effects parameters.


%-----------------------------------------------------------------
\subsection{NONMEM Mixed Effects Model}
%-----------------------------------------------------------------

NONMEM's mixed effects model 
\footnote{
  {\em NONMEM User's Guide - Part V}, NONMEM Project Group, 
  pp. 32-9, 1994.
} 
for the $j^{\mbox{th}}$ value of the $i^{\mbox{th}}$ individual's data is 
  \begin{equation}
    y_{i(j)} = f(x_{ij}, \theta, \eta_i) + 
      \sum_{m=1}^{n_{\Sigma}} h_{(m)}(x_{ij}, \theta, \eta_i) \epsilon_{i(m)} .
  \end{equation}
where
  \begin{equation}
    \Sigma_{(m,n)} = \mbox{cov}[\epsilon_{i(m)},\epsilon_{i(n)}] ,
  \end{equation}
  \begin{equation}
    \Omega_{(p,q)} = \mbox{cov}[\eta_{i(p)},\eta_{i(q)}] ,
  \end{equation}
$\theta$, $\Sigma$, and $\Omega$ are vectors and matrices of fixed effects parameters,
$\eta_i$ is the individual's vector of random effects parameters,
$x_{ij}$ is a vector of known quantities for the individual
such as times and covariates,
and $n_{\Sigma}$ is the number of nonzero elements in the lower 
triangle of $\Sigma$.

The output of NONMEM's population level estimation is
  \begin{equation}
    \left\{ \theta^{\mbox{Out}}, \Omega^{\mbox{Out}},
      \Sigma^{\mbox{Out}} \right\} ,
  \end{equation}
which are estimates for the optimal values for the fixed 
effects parameters.


%-----------------------------------------------------------------
\subsection{Comparison of the Mixed Effects Models}
%-----------------------------------------------------------------

In these two formulations, $b_i$ and $\eta_i$ are
equivalent.
Thus, the covariance matrices $D(\alpha)$ and 
$\Omega$ are equal to one another, and
  \begin{equation}
    D_{(p,q)}(\alpha) = \Omega_{(p,q)} .
  \end{equation}

Although $e_i$ and $\epsilon_i$ are not equivalent, 
the covariance matrices $R_i(\alpha, b_i)$ and 
$\Sigma$ are related as follows,
  \begin{eqnarray*}
    R_{i(j,k)}(\alpha, b_i) & = &  \mbox{cov} \left[
      \sum_{m=1}^{n_{\Sigma}} h_{(m)}(x_{ij}, \theta, \eta_i) \epsilon_{i(m)},
      \sum_{n=1}^{n_{\Sigma}} h_{(n)}(x_{ik}, \theta, \eta_i) \epsilon_{i(n)} \right] \\
    & = & \sum_{m=1}^{n_{\Sigma}} \sum_{n=1}^{n_{\Sigma}} 
      h_{(m)}(x_{ij}, \theta, \eta_i) 
      \mbox{cov} [ \epsilon_{i(m)} , \epsilon_{i(n)} ]
      h_{(n)}(x_{ik}, \theta, \eta_i)  \\
    & = & \sum_{m=1}^{n_{\Sigma}} \sum_{n=1}^{n_{\Sigma}} 
      h_{(m)}(x_{ij}, \theta, \eta_i) 
      \Sigma_{(m,n)}
      h_{(n)}(x_{ik}, \theta, \eta_i)
  \end{eqnarray*}
Note that $\Sigma$ is the same for all of the individuals in
the population.


%---------------------------------------------------------------
%
% Begin document
%
%---------------------------------------------------------------

\end{document}
