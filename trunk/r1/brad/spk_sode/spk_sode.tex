% LaTeX Article Template - customizing page format
%
% LaTeX document uses 10-point fonts by default.  To use
% 11-point or 12-point fonts, use \documentclass[11pt]{article}
% or \documentclass[12pt]{article}.
\documentclass{article}

% Set left margin - The default is 1 inch, so the following 
% command sets a 1.25-inch left margin.
\setlength{\oddsidemargin}{-0.25in}
\setlength{\evensidemargin}{-0.25in}

% Set width of the text - What is left will be the right margin.
% In this case, right margin is 8.5in - 1.25in - 6in = 1.25in.
\setlength{\textwidth}{7in}

% Set top margin - The default is 1 inch, so the following 
% command sets a 0.75-inch top margin.
\setlength{\topmargin}{-0.25in}

% Set height of the text - What is left will be the bottom margin.
% In this case, bottom margin is 11in - 0.75in - 9.5in = 0.75in
\setlength{\textheight}{8in}

\newcommand{\BE}{{\bf E}}
\newcommand{\BN}{{\bf N}}
\newcommand{\BR}{{\bf R}}
\newcommand{\RT}{{\rm T}}
\newcommand{\BV}{{\bf V}}
\newcommand{\BZ}{{\bf Z}}

\newcommand{\Bd}{{\bf d}}

\newcommand{\Th}[1]{{\( #1 \)}-th}

\newtheorem{lemma}{Lemma}

% Set the beginning of a LaTeX document
\begin{document}
\Large

\title{
	Nonlinear Mixed Effects Modeling with 
	Linear Stochastic Ordinary Differential Equations
}
\author{Bradley M. Bell}
\date{\today}
\maketitle


\section{Notation}
\begin{tabular}{lll}
Notation
	& Space
	& Description 
\\
M       & \( \BZ_+ \)
	& number of individuals in population study
\\
\( m \)
	& \( \BZ_+ \)
	& number of random effects per individual
\\
\( n \) 
	& \( \BZ_+ \)
	& number of state components in individual model 
\\
\( i \)
	& \( \BZ_+ \)
	& index indicating which individual  ( \( 1 \leq i \leq M \) )
\\
N(i)    & \( \BZ_+ \)
	& number of measurements 
\\
\( j \)
	& \( \BZ_+ \)
	& index indicating which measurement value  ( \( 1 \leq j \leq N(i) \) )
\\
\( t_j^i \)
	& \( \BR \)
	& measurement time
\\
\( y_j^i \)
	& \( \BR \)
	& measurement value
\\
\( \varepsilon^i \)
	& \( \BR^{N(i)} \)
	& measurement noise
\\
\( \Sigma^i ( \eta^i ) \)
	& \( \BR^{N(i) \times N(i)} \)
	& covariance of the measurement noise
\\
\( \eta^i \)
	& \( \BR^m \)
	& vector of random effects 
\\
\( \Omega \)
	& \( \BR^{m \times m} \)
	& variance of the random effects
\\
\( \beta^i (t , \eta^i ) \)
	& \( \BR^n \) 
	& state vector 
\\
\( d^i ( \eta^i ) \)
	& \( \BR^n \)
	& state vector at time \( t = 0 \)
\\
\( w^i (t) \)
	& \( \BR^n \) 
	& Weiner process 
\\
\( A^i  (  \eta^i )  \)
	& \( \BR^{n \times n} \)
	& state vector multiplier in stochastic ODE  
\\
\( B^i (  \eta^i ) \)
	& \( \BR^{n \times n} \)
	& Weiner process multiplier in stochastic ODE  
\\
\( C^i (  \eta^i ) \)
	& \( \BR^{N(i) \times n} \)
	& state vector multiplier in measurement equation
\end{tabular}

\bigskip
\noindent
The values
\( \Omega \),
\( \Sigma^i ( \eta^i ) \), 
\( A^i ( \eta^i ) \), 
\( B^i( \eta^i ) \), 
\( C^i ( \eta^i ) \), 
and
\( q( t, \eta^i ) \)
may depend on a fixed effects vector \( \alpha \).
Such a dependence would not
affect the results below
(but it is not included in order to simplify the presentation).

\bigskip
\noindent
We use \( \BV ( X , Y ) \) for the covariance of \( X \) and \( Y \); i.e.,
\begin{eqnarray*}
\BV ( X , Y ) & = & \BE \{ [ X - \BE (X) ] [ Y - \BE (Y) ]^\RT \}
\end{eqnarray*}

\bigskip
\noindent
We use the notation \( C^i_j ( \eta^i ) \) for the \Th{j}
row of the matrix \( C^i ( \eta^i ) \).

\section{Model Equations}
The model for the measurements is
\begin{eqnarray*}
y_j^i           & = &  
C^i_j (  \eta^i ) \beta^i ( t_j^i , \eta^i ) + \varepsilon_j^i
\\
\varepsilon^i   & \sim & \BN [ 0 , \Sigma^i ( \eta^i ) ]
\\
\eta^i          & \sim & \BN ( 0 , \Omega )
\end{eqnarray*}
where the random variables \( \{ \varepsilon_j^i \} \) and \( \{ \eta^i \} \)
are all mutually independent. 

\bigskip
\noindent
The stochastic differential equation that models 
the \Th{i} individual is
\begin{eqnarray*}
\beta^i (0, \eta^i ) 
& = & d^i ( \eta^i )
\\
\Bd \beta^i (t, \eta^i ) 
& = & 
A^i (  \eta^i ) \beta^i (t, \eta^i ) \Bd t + B^i (  \eta^i ) \Bd w^i (t)
\end{eqnarray*}
where the the random effects vector \( \eta^i \) is a fixed parameter and
the Weiner processes \( \{ w^i (t) \} \)
are all mutually independent and independent of the random variables
mentioned above. 


\section{Stochastic ODE}
We can write the stochastic differential equation above
as the following stochastic integral 
\begin{eqnarray*}
\beta^i ( t , \eta^i ) - \exp[ A^i (  \eta^i ) t ] d^i ( \eta^i ) 
& = &
\int_{0}^{t}
\exp[ A^i (  \eta^i ) ( t - \tau ) ] B^i (  \eta^i ) \Bd w^i ( \tau ) 
\end{eqnarray*}
It follows that the expected value of \( \beta^i (t, \eta^i ) \)
given \( \eta^i \) is
\begin{eqnarray*}
\BE [ \beta^i (t, \eta^i ) | \eta^i ]
& = &
\exp[ A^i (  \eta^i ) t ] d^i ( \eta^i )
\end{eqnarray*}
For \( t \geq s \), we define \( V^i (s , t | \eta^i ) \) 
to be the covariance variance of \( \beta^i (t, \eta^i ) \) 
and \( \beta^i (s, \eta^i ) \) given \( \eta^i \); i.e., 
\begin{eqnarray*}
V^i (s , t | \eta^i )
& = &
\BV [ \beta^i (s, \eta^i ) , \beta^i (t, \eta^i ) | \eta^i ]
\end{eqnarray*}
A direct application of the Ito isometry yields the following equation
\begin{eqnarray*}
V^i (t , t | \eta^i )
& = &
\int_0^{t}
\exp[ A^i (  \eta^i ) ( t - \tau ) ] 
	B^i (  \eta^i ) B^i (  \eta^i )^\RT 
		\exp[ A^i (  \eta^i ) ( t - \tau ) ]^\RT
			\Bd \tau
\end{eqnarray*}
It follows that this variance satisfies the following
intial value ordinary differential equation
\begin{eqnarray*}
V^i ( 0 ,  0 | \eta^i ) & = & 0
\\
\partial_t  V^i (t , t |  \eta^i )  
& = & A^i  (  \eta^i ) V^i (t , t | \eta^i ) 
  +   V^i (t , t | \eta^i ) A^i (  \eta^i )^\RT 
  +   B^i (  \eta^i ) B^i (  \eta^i )^\RT 
\end{eqnarray*}
It also follows from the independent increments in a Weiner process that
for \( t > s \)
\begin{eqnarray*}
V^i ( s , t | \eta^i ) & = & V^i (s , s | \eta^i )
\end{eqnarray*}
We define \( E^i (t | \eta^i ) \in \BR^n \) to be the expected value
of \( \beta^i (t, \eta^i ) \) given \( \eta^i \).
We also have the following initial value ordinary differential equaiton
for this expected value
\begin{eqnarray*}
E^i (0 | \eta^i ) & = & d^i ( \eta^i )
\\
\partial_t E^i (t | \eta^i ) & = & A^i  (  \eta^i ) E^i (t | \eta^i )
\end{eqnarray*}


\section{Likelihood Functions}
The combination of all the model random variables 
is multi-normally distributed.
In addition, except for functions of the fixed
and random effects, all the models are linear.
Thus, we can specify the likelihood for any random variable,
given the fixed and random effects, by determining its mean and variance.
We define \( F^i ( \eta^i ) \in \BR^{N(i)} \) as the mean of
\( y^i \) given \( \eta^i \)
\begin{eqnarray*}
F^i_j ( \eta^i )
& = &
\BE [ y_j^i | \eta^i ] 
\\
& = & 
C^i_j ( \eta^i ) E^i ( t_j^i | \eta^i )
\end{eqnarray*}
We use the notation
\( R_{j,k}^i ( \eta^i ) \in \BR^{N(i) \times N(i)} \) 
for the conditional covariance of \( y^i_j \) with \( y^i_k \)
which is given as follows:
for \( j \leq k \) ( \( t_j^i \leq t_k^i \) )
\begin{eqnarray*}
R_{j,k}^i ( \eta^i )
& = &
R_{k,j}^i ( \eta^i )
\\
& = &
\BV [ y_j^i , y_k^i | \eta^i , \beta^i ( 0 , \eta^i ) ]
\\
& = &
C^i_j ( \eta^i ) V^i ( t_j^i , t_k^i | \eta^i ) C^i_k ( \eta^i )^\RT
+
\Sigma^i_{j,k} ( \eta^i )
\end{eqnarray*}

\section{Example}
We now apply the analysis above to a particular example.
Suppose we are given a two compartment model diagrammed below.
For the moment, we
drop the dependence on time \( t \),
individual index \( i \),
and random effects value \( \eta^i \),
to simply the presentation.


% ===========================================================================
% The default \unitlength for a picture is one point; i.e., ~ 1/72 inches
% diagram with size (250, 200) and lower left corner at (-40,0)
% In the picture below 0  <= x <= 240, 40 <= y <= 175 
\begin{picture}(320,215)(-40,0)

% Bolus injection -----------------------------------------
% text d with its bottom center at location(15, 110)
\put(15, 110){\makebox(0,0)[bc]{\( d=1 \)}}

% line starting at (0, 105), dx = 1, dy = 0, length = 20
\put(0, 105){\line(1, 0){30}}

% line starting at (0,  95), dx = 1, dy = 0, length = 20
\put(0,  95){\line(1, 0){30}}

% line starting at (40, 100), dx = -1, dy = 1, length = 10
\put(40, 100){\line(-1, 1){10}}

% line starting at (40, 100), dx = -1, dy = -1, length = 10
\put(40, 100){\line(-1, -1){10}}

% First compartment --------------------------------------
% circle centered a (60, 100), diameter = 40
\put(60,100){\circle{40}}

% text q_0 centered at the location (60, 100)
\put(60, 100){\makebox(0,0)[c]{\( q_0 \)}}

% vector starting at (100, 100), dx = 1, dy = 0, length = 40
\put(80, 100){\vector(1,0){40}}

% text k_a with its bottom center at location (100, 105)
\put(100, 105){\makebox(0,0)[bc]{\( k_a \)}}

% Second compartment --------------------------------------
% circle centered a (140, 100), diameter = 40
\put(140,100){\circle{40}}

% text q_1 centered at the location (140, 100)
\put(140, 100){\makebox(0,0)[c]{\( q_1 \)}}

% vector starting at (160, 100), dx = 1, dy = 0, length = 40
\put(160, 100){\vector(1,0){40}}

% text k_e with its bottom center at location (200, 105)
\put(180, 105){\makebox(0,0)[bc]{\( k_e \)}}

% line starting at (155, 115), dx = 1, dy = 1, length = 40
\put(155, 115){\line(1,1){40}}

% text y = \beta q_1  + epsilon with bottom center at (195, 155)
\put(155, 155){
	\framebox(100,20)[c]{\( y = \beta q_1  + \varepsilon \)}
}

\end{picture}
% ===========================================================================

\bigskip
\noindent
We now include the dependence on time \( t \) and write
the corresponding deterministic ordinary differential equations
\[
\begin{array}{rclcrcl}
q_0 (0) & = & 1 & , & q_0 (t) & = & -k_a q_0 (t) 
\\ 
q_1 (0) & = & 0   & ,  & q_1 (t) & = & +k_a q_0 (t) - k_e q_1 (t) 
\end{array}
\]
The corresponding solution is
\begin{eqnarray*}
q_0 (t)   & = & \exp( - k_a t ) \\
q_1 (t)   & = & k_a \frac{ \exp( - k_a t ) - \exp( -k_e t ) }{ k_e - k_a }
\end{eqnarray*}
We now include the dependence on individual index \( i \)
random effects vector \( \eta^i \),
and fixed effects vector \( \alpha \) in our model:
\begin{eqnarray*}
\Bd \beta^i (t, \alpha, \eta^i ) & = & 
	\exp( \alpha_6 / 2 ) \; \Bd w (t)  
\\
\beta^i (0, \alpha, \eta^i ) & = & 
	\alpha_0 \exp( \eta^i_0 )   
\\
k_a ( \alpha , \eta^i ) & = & 
	\alpha_1 \exp( \eta^i_1 ) 
\\
k_e ( \alpha , \eta^i ) & = & 
	\alpha_2 \exp( \eta^i_2 ) 
\\
\Omega_{k, \ell} ( \alpha ) & = & \left\{ \begin{array}{ll}
	\exp( \alpha_{k + 3} ) & \mbox{if} \; k = \ell \in \{ 0 , 1 , 2 \} 
	\\
	0                      & \mbox{otherwise}
\end{array} \right.
\\
\Sigma^i_{j, k} & = & \left\{ \begin{array}{ll}
	\sigma^2 & \mbox{if} \; j = k \in \{ 1 , \ldots , N(i) \} 
	\\
	0                      & \mbox{otherwise}
\end{array} \right.
\end{eqnarray*}

\bigskip
\noindent
Given \( \alpha \) and \( \eta^i \) we can calculate the necessary SPK model 
function values for 
\( i = 1 , \ldots , M \),
\( j = 1 , \ldots , N(i) \),
\( k = 1 , \ldots , N(i) \),
\( \ell = 1 , \ldots , 4 \)
\begin{eqnarray*}
\Omega & = & \left( \begin{array}{ccc}
	\exp( \alpha_3 ) & 0 & 0 \\
	0 & \exp( \alpha_4 ) & 0 \\
	0 & 0 & \exp( \alpha_5 ) 
\end{array} \right)
\\
k_a^i & = & \alpha_1 \exp( \eta^i_1 ) 
\\
k_e^i & = & \alpha_2 \exp( \eta^i_2 ) 
\\
q^i_{1,j} & = & 
	k_a^i 
		\frac{ \exp( - k_a^i t^i_j ) - \exp( -k_e^i t^i_j ) } 
			{ k_e^i - k_a^i }
\\
F^i_j & = & q^i_{1,j} \alpha_0 \exp( \eta^i_0 ) 
\\
R^i_{j,k} & = & 
	\Sigma_{j,k}^i +
		\exp( \alpha_6 ) q^i_{1,j} \min\{ t_j , t_k \}  q^i_{1,k} 
\end{eqnarray*}

\end{document}
