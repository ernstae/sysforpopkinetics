% LaTeX Article Template - customizing page format
%
% LaTeX document uses 10-point fonts by default.  To use
% 11-point or 12-point fonts, use \documentclass[11pt]{article}
% or \documentclass[12pt]{article}.
\documentclass{article}

% Set left margin - The default is 1 inch, so the following 
% command sets a 1.25-inch left margin.
\setlength{\oddsidemargin}{-0.25in}
\setlength{\evensidemargin}{-0.25in}

% Set width of the text - What is left will be the right margin.
% In this case, right margin is 8.5in - 1.25in - 6in = 1.25in.
\setlength{\textwidth}{7in}

% Set top margin - The default is 1 inch, so the following 
% command sets a 0.75-inch top margin.
\setlength{\topmargin}{-0.25in}

% Set height of the text - What is left will be the bottom margin.
% In this case, bottom margin is 11in - 0.75in - 9.5in = 0.75in
\setlength{\textheight}{8in}

\newcommand{\BE}{{\bf E}}
\newcommand{\BN}{{\bf N}}
\newcommand{\BR}{{\bf R}}
\newcommand{\RT}{{\rm T}}
\newcommand{\BV}{{\bf V}}
\newcommand{\BZ}{{\bf Z}}

\newcommand{\Bd}{{\bf d}}
\newcommand{\Bp}{{\bf p}}

\newcommand{\Th}[1]{{\( #1 \)}-th}

\newtheorem{lemma}{Lemma}

% Set the beginning of a LaTeX document
\begin{document}
\Large

\title{
	Nonlinear Mixed Effects Modeling with 
	Linear Stochastic Ordinary Differential Equations
}
\author{Bradley M. Bell}
\date{\today}
\maketitle

\section{Individual Model}

% ===========================================================================
% The default \unitlength for a picture is one point; i.e., ~ 1/72 inches
% diagram with size (250, 200) and lower left corner at (-40,0)
% In the picture below 0  <= x <= 240, 40 <= y <= 175 
\begin{picture}(320,215)(-40,0)

% First compartment --------------------------------------
% circle centered a (60, 100), diameter = 40
\put(60,100){\circle{40}}

% text q(t) centered at the location (60, 100)
\put(60, 100){\makebox(0,0)[c]{\( q(t) \)}}

% vector starting at (100, 100), dx = 1, dy = 0, length = 40
\put(80, 100){\vector(1,0){40}}

% text a with its bottom center at location (100, 105)
\put(100, 105){\makebox(0,0)[bc]{\( a \)}}

% Second compartment --------------------------------------
% circle centered a (140, 100), diameter = 40
\put(140,100){\circle{40}}

% text r(t) centered at the location (140, 100)
\put(140, 100){\makebox(0,0)[c]{\( r(t) \)}}

% vector starting at (160, 100), dx = 1, dy = 0, length = 40
\put(160, 100){\vector(1,0){40}}

% text b with its bottom center at location (200, 105)
\put(180, 105){\makebox(0,0)[bc]{\( b \)}}

% line starting at (155, 115), dx = 1, dy = 1, length = 40
\put(155, 115){\line(1,1){40}}

% text y(t) = r(t)  + eps(t) with bottom center at (195, 155)
\put(155, 155){
	\framebox(140,20)[c]{\( y(t) = r(t)  + \varepsilon(t) \)}
}

\end{picture}
% ===========================================================================

\[
\begin{array}{rcllcl}
\partial_t q(t) & = & - a q(t)                     & , & q(0) = q_0  
\\
\partial_t r(t) & = & + a q(t) - b r(t)            & , & r(0) = r_0
\\
y(t)                      & = & r(t) + \varepsilon (t)
\end{array}
\]
The solution of the differential equation is
\begin{eqnarray*}
q(t) & = & q_0 \exp ( - a t )
\\
r(t) & = & \left[ 
	r_0 + a q_0 \frac{ 1  - \exp[ - (a - b) t ] }{ a - b } 
\right] \exp( - b t )
\end{eqnarray*}


\section{Stochastic Difference Equation}
Suppose we have a constant time points \( t_j \),
we write the corresponding stochastic difference equations:
\begin{eqnarray*}
a_j & = & a_{j-1} + \gamma_{j-1}
\\
b_j & = & b_{j-1} + \delta_{j-1}
\\
q_j & = & q_{j-1} \exp [ - a_{j-1} ( t_j - t_{j-1} ) ] 
\\ 
r_j & = & 
\left [ r_{j-1} + a_{j-1} q_{j-1} 
	\frac{ 1 - \exp[ - (a_{j-1} - b_{j-1}) ( t_j - t_{j-1} ) ] }
		{ a_{j-1} - b_{j-1} }
\right] \exp[ - b_{j-1} ( t_j - t_{j-1} ) ]
\\
y_j     & = & r_j + \varepsilon_j
\end{eqnarray*}
This is a naive representation of a stochastic differential equation that
is just meant for demonstration purposes.
For our example model,
the first two difference equations are deterministic
and the second two are stochastic as follows:
\( \varepsilon_j \sim \BN(0, \sigma_\varepsilon^2 ) \),
\( \gamma_j \sim \BN(0, \sigma_\gamma^2 ) \),
\( \delta_j \sim \BN(0, \sigma_\delta^2 ) \)
where the random variables are all independent.
We use the vector \( x_j \in \BR^4 \) to denote the state vector
\[
x_j = [ a_j , b_j, q_j , r_j ]^\RT
\]
Note that, given \( x_0 \), \( \gamma \) and \( \delta \)
we can use the difference equations above to calculate 
\(  q \), \( r \), \( a \) and \( b \).


\section{Population Mixed Effects Model}
We use \( \alpha \) for our fixed effects and \( \eta \)
for our random effects and \( i \) to index individuals in population.
The initial conditions \( x_0 \) are given by
\begin{eqnarray*}
q^i_0 & = & 1
\\
r^i_0 & = & 0
\\
a^i_0 & = & \alpha_0 \exp ( \eta^i_0 )
\\
b^i_0 & = & \alpha_1 \exp ( \eta^i_1 )
\end{eqnarray*}
The stochastic part of the differential equations are
given by random effects; to be specific, for \( j = 0 , \ldots , N-1 \)
\begin{eqnarray*}
\gamma^i_j & = & \eta^i_{2 + j}
\\
\delta^i_j & = & \eta^i_{N+1+j}
\end{eqnarray*}
All of the random variables listed below are independent:
\begin{eqnarray*}
\eta^i_0 & \sim & \BN [ 0, \exp( \alpha_2 ) ] 
\\
\eta^i_1 & \sim & \BN [ 0, \exp( \alpha_3 ) ] 
\\
\eta^i_{2+j} & \sim & \BN [ 0, \exp( \alpha_4 ) ] 
\\
\eta^i_{N+1+j} & \sim & \BN [ 0, \exp( \alpha_5 ) ] 
\\
\varepsilon^i_j & \sim & \BN [ 0 , \exp( \alpha_6 ) ]
\end{eqnarray*}

\end{document}
