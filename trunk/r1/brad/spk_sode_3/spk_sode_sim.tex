\documentclass{article}
\usepackage{epsfig}

\title{Using Stochastic ODE's in Compartmental Model Design}
\author{Bradley M. Bell}
\date{2007-09-10}

\newcommand{\B}[1]{{\bf #1}}
\newcommand{\R}[1]{{\rm #1}}

\begin{document}
\maketitle

\section{Abstract}
We present a simulated example application of a method 
that uses stochastic ordinary differential equations 
as a tool in the design of population compartmental PK models.
The example individual model has only two transfer rates;
its absorption and elimination rates.
Data is simulated with one rate constant
and the other changing with respect to time.
Both the changing absorption rate
and changing elimination rate cases are simulated.
The models are then fit with one rate constant and the 
other rate stochastic with respect to time.
The fitting results in much better likelihood values
(lower values for the negative log-likelihood objective)
when the time-varying rate is modeled as stochastic.
We also check that, when both rates are constant in time,
both stochastic models detect this fact.

\section{Compartmental Model}
We are given a maximum simulation time \( T \).
Our compartmental model 
is parametrized by two stochastic functions
\( a (t) \) and \( b (t) \) for \( t \in [0, T] \).
The corresponding ordinary differential equation for the
concentration in the absorption compartment \( q(t) \)
and in the central compartment \( r(t) \) is
\begin{eqnarray*}
q (0)  & = & q_0
\\
q' (t) & = & - a(t) q(t)
\\
r (0) & = & r_0
\\
r' (t) & = & + a(t) q(t) - b(t) r(t)
\end{eqnarray*}
where \( q_0 \) and \( r_0 \) are the initial concentrations
in the corresponding compartments.
The solution of this ODE is
\begin{eqnarray*}
q ( t ) 
& = & 
q_0 \exp \left[ - \int_0^t a(s) \B{d} s \right]
\\
r ( t ) 
& = & 
r_0 \exp \left[ - \int_0^t b(s) \B{d} s \right]
\\
& + & 
\int_0^t \exp \left[ 
	\int_0^s b(u) \B{d} u - \int_0^t  b(u) \B{d} u 
\right] a(s) q(s) \B{d} s
\end{eqnarray*}
For general stochastic differential equations the functions
\( a(t) \) and \( b(t) \) would be expressed in terms of 
Brownian motion.

\section{Example Rate Functions}
We consider a very simple approximation where the rate functions
\( a(t) \) and \( b(t) \) are piecewise constant.
This enables us to use existing mixed effects modeling programs
to test our method for designing compartmental models.
To be specific, 
we fix a set of times \( t_0 < t_q < \cdots < t_N \)
where \( t_0 = 0 \) and \( t_N = T \),
corresponding absorption rate values \( a_0 , \ldots , a_{N-1} \),
and elimination rate values \( b_0 , \ldots , b_{N-1} \).
The functions \( a(t) \) and \( b(t) \) are defined for 
\( t \in [0, T) \) by 
\begin{eqnarray*}
a (t) & = & a_j \; \R{for} \; t \in [ t_j , t_{j+1} )
\\
b (t) & = & b_j \; \R{for} \; t \in [ t_j , t_{j+1} )
\end{eqnarray*}

\noindent
We are given \( q_0 \) and \( r_0 \) and for \( j = 1 , \ldots , N \)
we use \( q_j \) and \( r_j \)
to denote \( q ( t_j ) \) and \( r ( t_j ) \) respectively.
It follows from the equations for \( q(t) \) and \( r(t) \) above that
we can express \( q_{j+1} \) and \( r_{j+1} \) in terms of \( q_j \)
and \( r_j \) as follows:
\begin{eqnarray*}
q_{j+1} 
& = & 
q_j \exp [ - a_j ( t_{j+1} - t_j ) ] 
\\
r_{j+1} 
& = & 
r_j \exp [ - b_j ( t_{j+1} - t_j ) ]
\\
& + &
\int_{t(j)}^{t(j+1)}
	\exp [ b_j ( s - t_j ) - b_j ( t_{j+1} - t_j ) ] 
		a_j q_j  \exp [ - a_j ( s - t_j ) ]
			\B{d} s
\\
& = &
r_j \exp [ - b_j ( t_{j+1} - t_j ) ]
\\
& + &
a_j q_j \exp [ - b_j ( t_{j+1} - t_j ) \exp [ (a_j - b_j ) t_j ] 
	\int_{t(j)}^{t(j+1)}
		\exp[ ( b_j - a_j ) s ]  
			\B{d} s
\\
& = &
\exp [ - b_j ( t_{j+1} - t_j ) ] \left\{ 
r_j
+
a_j q_j \exp [ (a_j - b_j ) t_j ]  
\frac{ \exp [ ( b_j - a_j ) t_{j+1} - \exp[ ( b_j - a_j ) t_j ]}{b_j - a_j}
\right\}
\\
& = &
\exp [ - b_j ( t_{j+1} - t_j ) ] \left\{ 
r_j
+
a_j q_j 
\frac{ \exp [ ( b_j - a_j ) ( t_{j+1} - t_j ) ] - 1 }{b_j - a_j}
\right\}
\\
& = &
\exp [ - b_j ( t_{j+1} - t_j ) ] \left\{ 
r_j
+
a_j q_j 
\frac{ 1 - \exp [ - ( a_j - b_j ) ( t_{j+1} - t_j ) ] }{a_j - b_j}
\right\}
\end{eqnarray*}
We now define the functions that relate \( q_{j+1} \) and \( r_{j+1} \)
to \( q_j \) and \( r_j \);
\begin{eqnarray*}
Q_{j+1} ( a_j , b_j , q_j , r_j ) 
& = &
q_j \exp [ - a_j ( t_{j+1} - t_j ) ] 
\\
R_{j+1} ( a_j , b_j , q_j , r_j ) 
& = &
\exp [ - b_j ( t_{j+1} - t_j ) ] \left\{ 
r_j
+
a_j q_j 
\frac{ 1 - \exp [ - ( a_j - b_j ) ( t_{j+1} - t_j ) ] }{a_j - b_j}
\right\}
\end{eqnarray*}

\section{Simulating Data}
When simulating the data we used the following constants

\bigskip
\begin{tabular}{lll}
{\bf Notation}    & {\bf Value} & {\bf Description} \\
\( M \)           &  20     & number of individuals in population data set \\
\( N \)           &  5      & number of measurement per individual \\
\( T \)           &  1      & time corresponding to last measurement \\
\( q_0 \)         &  1      & initial absorption compartment concentration \\
\( r_0 \)         &  0      & initial central compartment concentration  \\
\( \mu_a \)       &  2      & mean initial absorption rate \( a_0 \) \\
\( \mu_b \)       &  2      & mean initial absorption rate \( b_0 \) \\
\( \sigma_a \)    &  .1     & standard deviation for \( a_0 \) \\
\( \sigma_b \)    &  .1     & standard deviation for \( a_0 \) \\
\( \Delta a \)    & 0, 0, .2  & change in absorption rate \( a_j - a_{j-1} \) \\
\( \Delta b \)    & 0, .2, 0  & change in elimination rate \( a_j - a_{j-1} \) 
\end{tabular}

\bigskip
\noindent
Note that there are three simulated data sets corresponding to the 
three pairs of values for \( ( \Delta a , \Delta b ) \).
For \( i = 0 , \ldots , M-1 \)
and \( j = 0 , \ldots , N-1 \)
we simulate independent values 
\begin{eqnarray*}
\eta_{i,0} & \sim & \B{N} (0, \sigma_a^2 ) \\
\eta_{i,1} & \sim & \B{N} (0, \sigma_b^2 ) \\
\varepsilon_{i,j} & \sim & \B{N} (0, \sigma_\varepsilon^2 ) 
\end{eqnarray*}
We then compute the simulated values
\begin{eqnarray*}
a_{i,0} & = & \mu_a + \eta_{i,0}        \\
b_{i,0} & = & \mu_b + \eta_{i,1}        \\
a_{i,j} & = & a_{i, j-1} + \Delta a     \\
b_{i,j} & = & b_{i, j-1} + \Delta b     \\
q_{i, j+1} & = & Q_{j+1} ( a_j , b_j , q_j , r_j ) 
\\
r_{i, j+1} & = & R_{j+1} ( a_j , b_j , q_j , r_j ) 
\\
y_{i,j} & = & r_{i, j+1} + \varepsilon_{i,j}
\end{eqnarray*}


\section{Mixed Effects Model}
In this section we define the nonlinear mixed effects model
that is used to fit the data (using {\sc spk} or some other 
nonlinear mixed effects modeling program; e.g., {\sc nonmem}).
Parameters are known values 
(do not need to be estimated) that are the same for all individuals
and all measurement values:

\bigskip
\begin{tabular}{lll}
{\bf Parameter} & {\bf Value} & {\bf Description} 
\\
\( M \)        & 10
& number of individuals in simulated population
\\
\( N \)        & 10
& number of measurements per individual
\\
\( T \)        & 1
& time corresponding to last measurement
\\
\( \lambda \)  & 0 , 1
& zero (one) if \( a_{i,j} \) ( \( b_{i,j} \) ) constant w.r.t \( i \)
\\
\( q_0 \)      & 1
& concentration in absorption compartment at time zero
\\
\( r_0 \)     & 0
& concentration in central compartment at time zero
\\
\( \mu_a \)   & 2
& mean value for \( a_{i,0} \)
\\
\( \mu_b \)   & .5
& mean value for \( b_{i,0} \)
\\
\( \sigma_\varepsilon \) & .001
& standard deviation of measurement noise
\end{tabular}

\bigskip
\noindent
Fixed effects are unknown values 
(need to be estimated) that are the same for all individuals
and all measurement values
(note that some of the fixed effects have been log scaled to improve
the optimization procedure):

\bigskip
\begin{tabular}{ll}
{\bf Fixed Effect} & {\bf Description}
\\
\( \log ( \sigma_a^2 ) \)    & log of variance of \( \eta_{i,0} \) 
\\
\( \log ( \sigma_b^2 ) \)    & log of variance of \( \eta_{i,1} \) 
\\
\( \log ( \sigma_\eta^2 ) \) & log of variance of change in transfer rate
\end{tabular}


\bigskip
\noindent
Random effects are unknown random values that are independently distributed
between individuals 
and are the same for all measurements corresponding to one individual.
Note that the model below is unusual in that there is a random
effect corresponding to each measurement index.
Each of these random variables is independent from the other random variables.
We use \( i \) to represent the dependence on individual 
in the table below:

\bigskip
\begin{tabular}{lll}
{\bf Random Effect}  & {\bf Distribution} & {\bf Description}
\\
\( \eta_{i,0} \) &  \( \B{N} ( 0 , \sigma_a^2 ) \)    
& initial absorption rate is \( \mu_a + \eta_{i,0} \)
\\
\( \eta_{i,1} \) & \( \B{N} ( 0 , \sigma_b^2 ) \)    
& initial elimination rate if \( \mu_b + \eta_{i,1} \)
\\
\( \eta_{i,j} \) & \( \B{N} ( 0 , \sigma_\eta^2 ) \)    
& change in elimination rate (for \( j \geq 2 \) )
\end{tabular}

\bigskip
\noindent
We are given the initial concentrations 
\( q_{i,0} = q_0 \), \( r_{i,0} = r_0 \).
Our stochastic model for the rate constants \( a_i (t) \) and \( b_i (t) \)
is specified by the following equations for
\( i = 0 , \ldots , M-1 \) and \( j = 1 , \ldots , N-1 \):
\begin{eqnarray*}
a_{i,0} & = & \mu_a + \eta_{i,0} \\
b_{i,0} & = & \mu_b + \eta_{i,1} \\
a_{i,j} & = & a_{i, j-1} + \lambda \eta_{i, j+1} \\
b_{i,j} & = & b_{i, j-1} + (1 - \lambda ) \eta_{i, j+1} \\
a_i (t) & = & a_{i,j} \; \R{for} \; t \in [ t_j , t_{j+1} ) \\
b_i (t) & = & b_{i,j} \; \R{for} \; t \in [ t_j , t_{j+1} )
\end{eqnarray*}

\bigskip
\noindent
The measurement noise \( \{ \varepsilon_{i,j} \} \)
is a set of unknown random values 
that are independent from the other random variables above and
from each other.
The mixed effects model for the the concentration \( \{ r_{i, j+1} \} \),
and the measurement values \( \{ y_{i,j} \} \)
for \( i = 0 , \ldots , M-1 \)
and \( j = 0 , \ldots , N-1 \) is given by
\begin{eqnarray*}
q_{i, j+1} & = & Q_{j+1} ( a_j , b_j , q_j , r_j ) 
\\
r_{i, j+1} & = & R_{j+1} ( a_j , b_j , q_j , r_j ) 
\\
y_{i,j} & = & r_{i, j+1} + \varepsilon_{i,j}
\end{eqnarray*}
where the values \( a_{i,j} \) and \( b_{i,j} \) are related
to the the random variables  \( \{ \eta_{i,j} \} \) by the equations above.

\section{Results}
% Plot file naming convention: plot_lambda_deltaA_deltaB.eps
%
\subsection{Simulated With Absorption Rate Varying in Time}
\epsfig{file=plot_0_0.2_0.eps}
\epsfig{file=plot_1_0.2_0.eps}
%
\subsection{Simulated With Elimination Rate Varying in Time}
\epsfig{file=plot_0_0_0.2.eps}
\epsfig{file=plot_1_0_0.2.eps}
%
\subsection{Simulated With Both Rates Constant in Time}
\epsfig{file=plot_0_0_0.eps}
\epsfig{file=plot_1_0_0.eps}


\end{document}
