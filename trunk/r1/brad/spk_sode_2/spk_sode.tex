% LaTeX Article Template - customizing page format
%
% LaTeX document uses 10-point fonts by default.  To use
% 11-point or 12-point fonts, use \documentclass[11pt]{article}
% or \documentclass[12pt]{article}.
\documentclass{article}

% Set left margin - The default is 1 inch, so the following 
% command sets a 1.25-inch left margin.
\setlength{\oddsidemargin}{-0.25in}
\setlength{\evensidemargin}{-0.25in}

% Set width of the text - What is left will be the right margin.
% In this case, right margin is 8.5in - 1.25in - 6in = 1.25in.
\setlength{\textwidth}{7in}

% Set top margin - The default is 1 inch, so the following 
% command sets a 0.75-inch top margin.
\setlength{\topmargin}{-0.25in}

% Set height of the text - What is left will be the bottom margin.
% In this case, bottom margin is 11in - 0.75in - 9.5in = 0.75in
\setlength{\textheight}{8in}

\newcommand{\BE}{{\bf E}}
\newcommand{\BN}{{\bf N}}
\newcommand{\BR}{{\bf R}}
\newcommand{\RT}{{\rm T}}
\newcommand{\BV}{{\bf V}}
\newcommand{\BZ}{{\bf Z}}

\newcommand{\Bd}{{\bf d}}
\newcommand{\Bp}{{\bf p}}

\newcommand{\Th}[1]{{\( #1 \)}-th}

\newtheorem{lemma}{Lemma}

% Set the beginning of a LaTeX document
\begin{document}
\Large

\title{
	Nonlinear Mixed Effects Modeling with 
	Linear Stochastic Ordinary Differential Equations
}
\author{Bradley M. Bell}
\date{\today}
\maketitle

\section{Individual Model}

% ===========================================================================
% The default \unitlength for a picture is one point; i.e., ~ 1/72 inches
% diagram with size (250, 200) and lower left corner at (-40,0)
% In the picture below 0  <= x <= 240, 40 <= y <= 175 
\begin{picture}(320,215)(-40,0)

% First compartment --------------------------------------
% circle centered a (60, 100), diameter = 40
\put(60,100){\circle{40}}

% text q(t) centered at the location (60, 100)
\put(60, 100){\makebox(0,0)[c]{\( q(t) \)}}

% vector starting at (100, 100), dx = 1, dy = 0, length = 40
\put(80, 100){\vector(1,0){40}}

% text a with its bottom center at location (100, 105)
\put(100, 105){\makebox(0,0)[bc]{\( a \)}}

% Second compartment --------------------------------------
% circle centered a (140, 100), diameter = 40
\put(140,100){\circle{40}}

% text r(t) centered at the location (140, 100)
\put(140, 100){\makebox(0,0)[c]{\( r(t) \)}}

% vector starting at (160, 100), dx = 1, dy = 0, length = 40
\put(160, 100){\vector(1,0){40}}

% text b with its bottom center at location (200, 105)
\put(180, 105){\makebox(0,0)[bc]{\( b \)}}

% line starting at (155, 115), dx = 1, dy = 1, length = 40
\put(155, 115){\line(1,1){40}}

% text y(t) = r(t)  + eps(t) with bottom center at (195, 155)
\put(155, 155){
	\framebox(140,20)[c]{\( y(t) = r(t)  + \varepsilon(t) \)}
}

\end{picture}
% ===========================================================================

\[
\begin{array}{rcllcl}
\partial_t q(t) & = & - a q(t)                     & , & q(0) = q_0  
\\
\partial_t r(t) & = & + a q(t) - b r(t)            & , & r(0) = r_0
\\
y(t)                      & = & r(t) + \varepsilon (t)
\end{array}
\]
The solution of the differential equation is
\begin{eqnarray*}
q(t) & = & q_0 \exp ( - a t )
\\
r(t) & = & \left[ 
	r_0 + a q_0 \frac{ 1  - \exp[ - (a - b) t ] }{ a - b } 
\right] \exp( - b t )
\end{eqnarray*}


\section{Stochastic Difference Equation}
Suppose we have a constant time points \( t_j \),
we write the corresponding stochastic difference equations:
\begin{eqnarray*}
a_j & = & a_{j-1} + \gamma_{j-1}
\\
b_j & = & b_{j-1} + \delta_{j-1}
\\
q_j & = & q_{j-1} \exp [ - a_{j-1} ( t_j - t_{j-1} ) ] 
\\ 
r_j & = & 
\left [ r_{j-1} + a_{j-1} q_{j-1} 
	\frac{ 1 - \exp[ - (a_{j-1} - b_{j-1}) ( t_j - t_{j-1} ) ] }
		{ a_{j-1} - b_{j-1} }
\right] \exp[ - b_{j-1} ( t_j - t_{j-1} ) ]
\\
y_j     & = & r_j + \varepsilon_j
\end{eqnarray*}
This is a naive representation of a stochastic differential equation that
is just meant for demonstration purposes.
For our example model,
the first two difference equations are deterministic
and the second two are stochastic as follows:
\( \varepsilon_j \sim \BN(0, \sigma_\varepsilon^2 ) \),
\( \gamma_j \sim \BN(0, \sigma_\gamma^2 ) \),
\( \delta_j \sim \BN(0, \sigma_\delta^2 ) \)
where the random variables are all independent.
We use the vector \( x_j \in \BR^4 \) to denote the state vector
\[
x_j = [ a_j , b_j, q_j , r_j ]^\RT
\]
Note that, given \( x_0 \), \( \gamma \) and \( \delta \)
we can use the difference equations above to calculate 
\(  q \), \( r \), \( a \) and \( b \).
The negative log likelihood joint likelihood of the 
data and the model parameters is
\begin{eqnarray*}
- \log \Bp (y | \gamma , \delta, x_0, \sigma ) & = &
\frac{1}{2} \sum_{j=0}^{N-1} 
\log( 2 \pi \sigma_\varepsilon^2 )
+
\frac{ (y_j - r_j )^2 }{ \sigma_\varepsilon^2 } 
\\
& + &
\sum_{j=1}^{N-1}
\log( 2 \pi \sigma_\gamma^2 )
+
\frac{ \gamma_{j-1}^2 }{ \sigma_\gamma^2 } 
+
\log( 2 \pi \sigma_\delta^2 )
+
\frac{ \delta_{j-1}^2 }{ \sigma_\delta^2 } 
\end{eqnarray*}
Given \( x_0 \) and \( \sigma \),
the maximum likelihood estimate for \( \gamma \) and \( \delta \)
solves the problem
\[
\mbox{maximize} 
\Bp (y | \gamma , \delta, x_0, \sigma ) 
\; \mbox{w.r.t} \; \gamma , \delta 
\]
We use the notation
\[
[ \hat{\gamma} ( x_0 , \sigma ) , \hat{\delta} ( x_0 , \sigma ) ]
\]
to denote the solution of this problem.
In general, for larger models and for lots a data points,
the Kalman-Bucy smoother is used to solve this problem.
For our example case, we solve it by directly 
applying the Gauss-Newton method.
Calculation of the minimum variance estimate for \( \gamma \)
and \( \delta \) requires Markov chain Monte Carlo methods
(we are currently writing a paper about this).
By the law of total probability,
the likelihood of the data given \( x_0 \) and \( \sigma \) is
\[
\Bp ( y | x_0 , \sigma ) = 
\int \Bp ( y | \gamma , \delta , x_0 , \sigma )
~ \Bd \gamma ~ \Bd \delta
\]
In general, an Laplace like approximation or Markov chain Monte Carlo
methods can be used to approximate this integral
(we have written papers about using the Kalman-Bucy decomposition and
obtaining efficient and numerically stable computation of Laplace like 
approximations to this integral).
For this example, we use the maximum likelihood estimates
for \( \gamma \) and \( \delta \) to obtain an approximate objective
\[
\tilde{ \Bp } ( y | x_0 , \sigma ) =
\Bp [ y | 
	\hat{\gamma} ( x_0 , \sigma ) , 
	\hat{\delta} ( x_0 , \sigma ) , 
	x_0 , 
	\sigma 
]
\]
We realize that this has not properly corrected for the degrees of freedom
and that approximating the integral will fix this problem.


\section{Population Mixed Effects Model}
We use \( \alpha \) for our fixed effects and \( \eta \)
for our random effects and \( i \) to index individuals in population.
\begin{eqnarray*}
q^i_0 & = & 1
\\
r^i_0 & = & 0
\\
a^i_0 & = & \alpha_0 \exp ( \eta^i_0 )
\\
b^i_0 & = & \alpha_1 \exp ( \eta^i_1 )
\\
\eta^i & \sim&  \BN [ 0 , \Omega ( \alpha ) ]
\\
\Omega ( \alpha ) & = & \left( \begin{array}{cc}
	\exp( \alpha_2 ) & 0 \\
	0                & \exp( \alpha_3 )
\end{array} \right)
\\
\sigma_\varepsilon^2 & = & \exp( \alpha_4 )
\\
\sigma_\gamma^2      & = & \exp( \alpha_5 )
\\
\sigma_\delta^2      & = & \exp( \alpha_6 )
\end{eqnarray*}
Note that \( \sigma \) is a function of \( \alpha \),
\( x^i_0 \) is a function of \( \alpha \),  \( \eta^i \),
and the model above defines
\( \Bp ( \eta^i | \alpha ) \).
The joint likelihood of \( y^i \) and \( x^i_0 \) is
\begin{eqnarray*}
\Bp ( y^i , x^i_0 | \alpha )
& = &
\Bp ( y^i | x^i_0 , \sigma ) \Bp ( x^i_0 | \alpha )
\\
\tilde{ \Bp } ( y^i , x^i_0 | \alpha )
& = &
\tilde{ \Bp } ( y^i | x^i_0 , \sigma ) \Bp ( x^i_0 | \alpha )
\\
& = & 
\tilde{ \Bp } ( y^i | \eta^i , \alpha ) \Bp ( \eta^i | \alpha )
\end{eqnarray*}
We now use standard nonlinear mixed effects modeling methods to
approximate the integral
\[
\tilde{ \Bp } ( y^i | \alpha )
=
\int 
\tilde{ \Bp } ( y^i | \eta^i , \alpha ) \Bp ( \eta^i | \alpha ) ~ \Bd \eta^i
\]
For this example,
we present the estimate for \( \hat{\alpha} \) that maximizes
one of the standard approximations for the integral above; i.e.,
an approximate maximum likelihood estimate.
There is one caveat, 
we did not actually include
the log terms for \( \sigma_\gamma \) and \( \sigma_\delta \) in
example program, so we could not fit these fixed effects and had
to set them to a predefined value.
We also present the post-hock average weighted residual squared
for each individual; i.e.,
\begin{eqnarray*}
\bar{\varepsilon}^i & = & 
\sqrt{
\frac{1}{N} \sum_{j=1}^N 
\frac{ (y_j - \hat{r}_j )^2 }{ \hat{\sigma}_\varepsilon^2 } 
}
\\
\bar{\gamma}^i & = & 
\sqrt{
\frac{1}{N-1} \sum_{j=1}^{N-1} 
\frac{ \hat{\gamma}_{j-1}^2 }{ \sigma_\gamma^2 } 
}
\\
\bar{\delta}^i & = & 
\sqrt{
\frac{1}{N-1} \sum_{j=1}^{N-1} 
\frac{ \hat{\delta}_{j-1}^2 }{ \sigma_\delta^2 } 
}
\end{eqnarray*}
One could also use Monte-Carlo methods to obtain the minimum variance
estimate and or confidence intervals for these parameters.


\end{document}
